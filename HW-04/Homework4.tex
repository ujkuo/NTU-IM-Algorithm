\documentclass[a4paper,12pt]{article}
\usepackage[top = 2.5cm, bottom = 2.5cm, left = 2.5cm, right = 2.5cm]{geometry} 
\usepackage[T1]{fontenc}
\usepackage[utf8]{inputenc}
\usepackage{multirow} % Multirow is for tables with multiple rows within one cell.
\usepackage{booktabs} % For even nicer tables.

% As we usually want to include some plots (.pdf files) we need a package for that.
\usepackage{graphicx} 
\usepackage{amsmath} % to use split function

% The default setting of LaTeX is to indent new paragraphs. This is useful for articles. But not really nice for homework problem sets. The following command sets the indent to 0.
\usepackage{setspace}
\setlength{\parindent}{0in}

% Package to place figures where you want them.
\usepackage{float}
\usepackage{algorithm} 
\usepackage{algpseudocode} 
% The fancyhdr package let's us create nice headers.
\usepackage{fancyhdr}
\pagestyle{fancy} % With this command we can customize the header style.

\fancyhf{} % This makes sure we do not have other information in our header or footer.

\lhead{\footnotesize Algorithms: Homework 4}% \lhead puts text in the top left corner. \footnotesize sets our font to a smaller size.

%\rhead works just like \lhead (you can also use \chead)
\rhead{\footnotesize Yu-Chieh Kuo} %<---- Fill in your lastnames.

% Similar commands work for the footer (\lfoot, \cfoot and \rfoot).
% We want to put our page number in the center.
\cfoot{\footnotesize \thepage} 

\begin{document}
\thispagestyle{plain} % This command disables the header on the first page. 

\begin{tabular}{p{15.5cm}} % This is a simple tabular environment to align your text nicely 
{\large \bf The Design and Analysis of Algorithms} \\
National Taiwan University, Fall 2020  \\
\hline % \hline produces horizontal lines.
\\
\end{tabular} % Our tabular environment ends here.

\vspace*{0.3cm} % Now we want to add some vertical space in between the line and our title.

\begin{center} % Everything within the center environment is centered.
	{\Large \bf Homework 4} % <---- Don't forget to put in the right number
	\vspace{2mm}
	
        % YOUR NAMES GO HERE
	{\bf Yu-Chieh Kuo B07611039} % <---- Fill in your names here!
		
\end{center}  
\vspace{0.4cm}

This homework answers the problem set sequentially. 

\begin{enumerate}

%problem 1
\item { Let $K$ be the size of the knapsack, $S[i]$ be the size of the $i$-th item, and $P(n,K)$ be the solution with the number of items $n$ and the size of the knapsack $K$. The algorithm can be rewrite as below.

\begin{algorithm}
	\caption{ModifyKnapsack(K,S,P)} 
	\begin{algorithmic}[1]
	\State Let int k := K, Answer := []
	\For {$i := n \ldots 1$}
		\If{$P[i,k].exist = false$}
		        \State \textbf{return} 0
		\EndIf
		\If{$P[i,k].belong = true$}
		        \State $Answer.append(S[i])$
		        \State $k := K - S[i]$
		\EndIf
		\EndFor
		\If{$k \neq 0$}
		    \State \textbf{return} 0
		\EndIf
	\State \textbf{return} Answer
	\end{algorithmic} 
\end{algorithm}
}
\newpage
%problem 2
\item{ With the description of the statement, we can rewrite the knapsack algorithm as below.
\begin{algorithm}
	\caption{Knapsack(K,S,P)} 
	\begin{algorithmic}[1]
	\State Let $P[0,0].exist := true$
	\State Let $P[0,0].belong := 0$
	\For{$k := 1 \ldots K$}
		\State $P[0,k].exist := false$
	\EndFor
	\For{$i := 1 \ldots n$}
	    \For{$k := 0 \ldots K$}
	        \State $P[i,k].exist := false$
	        \If{$P[i-1,k].exist$}
	            \State $P[i,k].exist := true$
	            \State $P[i,k].belong := 0$
	        \ElsIf{$k-S[i] \geq 0$}
	            \If{$P[i, k-S[i]].exist$}
	                \State $P[i,k].exist := true$
	                \State $P[i,k].belong := P[i, k - S[i]].belong + 1$
	            \EndIf
	        \EndIf
	    \EndFor
	\EndFor
	\end{algorithmic} 
\end{algorithm}

}
\newpage
%problem 3
\item { Let $n$ be the size of a set of integers, $X[]$ be the set of integers, $S$ be the sum of the set, the algorithm can be represented as below.
\begin{algorithm}
	\caption{FindPartition(X[],n,S)} 
	\begin{algorithmic}[1]
	\If{$S$ is odd}
	    \State \textbf{return false}
	\EndIf
	\State Let int $s := \frac{S}{2}$, $dp[n][s]$, $set1, set2 = []$
	\If{$dp[n][s].exist$}
	    \While{ $n > 0$}
	        \If{$dp[n][s].belong$}
	            \State $set1.append(X_n)$
	            \State $s := s - x_n$
	        \Else
	            \State $set2.append(X_n)$
	        \EndIf
	        \State $n := n-1$
	   \EndWhile
	\Else
	    \State \textbf{return} false
    \end{algorithmic} 
\end{algorithm}
}


%problem 4
\item{
\begin{enumerate}
    \item {
The algorithm can be write as below.

\begin{algorithm}
	\caption{HanoiTower(n,A,B,C)} 
	\begin{algorithmic}[1]
	\If{$n := 1$}
	    \State move from A to B
	\ElsIf{$n > 1$}
	    \State \textbf{HanoiTower}(n-1,A,C,B)
	    \State move the largest disk from A to B
	    \State \textbf{HanoiTower}(n-1,C,B,A)
    \end{algorithmic} 
\end{algorithm}
To Explain how induction works here, we can use the inductive method. In the base case, we have 1 disk and we will move it from A to B. Assume that moving $n-1$ disks is available, when moving $n$ disks, we will move $n-1$ disks from A to C, move the largest disk to B, then move $n-1$ disk from C to B. 
    }
    \item {
Let $S(n)$ be the total steps of moves for Hanoi Tower, we have $S(1)=1$ and $S(n) = 2 S(n-1) + 1$. With easy computation, we can find $S(n) = 2^n - 1$.
    
    }
\end{enumerate}

}
\newpage
%problem 5
\item{
Without the recursive step, we can also write the algorithm to deal with the Hanoi Puzzle problem. The algorithm can be written as below. At first, we need to define a function \textit{move(m,n)} to make the legal move between two legs \textit{m,n} and print the move.
\begin{algorithm}
	\caption{NonRecursiveHanoiPuzzle(n,A,B,C)} 
	\begin{algorithmic}[1]
	\State Let $step := 0$
	\While{$step < 2^n - 1$}
	    \If{$n$ is even}
	        \State move(A,C)
	        \State move(A,B)
	        \State move(C,B)
	   \Else
	        \State move(A,B)
	        \State move(A,C)
	        \State move(C,B)
	   \EndIf
	   \State $step := step + 1$
	   \EndWhile
\end{algorithmic} 
\end{algorithm}
}

\end{enumerate}

\end{document}
