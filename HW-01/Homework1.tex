\documentclass[a4paper,12pt]{article}
\usepackage[top = 2.5cm, bottom = 2.5cm, left = 2.5cm, right = 2.5cm]{geometry} 
\usepackage[T1]{fontenc}
\usepackage[utf8]{inputenc}
\usepackage{multirow} % Multirow is for tables with multiple rows within one cell.
\usepackage{booktabs} % For even nicer tables.

% As we usually want to include some plots (.pdf files) we need a package for that.
\usepackage{graphicx} 
\usepackage{amsmath} % to use split function

% The default setting of LaTeX is to indent new paragraphs. This is useful for articles. But not really nice for homework problem sets. The following command sets the indent to 0.
\usepackage{setspace}
\setlength{\parindent}{0in}

% Package to place figures where you want them.
\usepackage{float}

% The fancyhdr package let's us create nice headers.
\usepackage{fancyhdr}
\pagestyle{fancy} % With this command we can customize the header style.

\fancyhf{} % This makes sure we do not have other information in our header or footer.

\lhead{\footnotesize Algorithms: Homework 1}% \lhead puts text in the top left corner. \footnotesize sets our font to a smaller size.

%\rhead works just like \lhead (you can also use \chead)
\rhead{\footnotesize Yu-Chieh Kuo} %<---- Fill in your lastnames.

% Similar commands work for the footer (\lfoot, \cfoot and \rfoot).
% We want to put our page number in the center.
\cfoot{\footnotesize \thepage} 

\begin{document}
\thispagestyle{empty} % This command disables the header on the first page. 

\begin{tabular}{p{15.5cm}} % This is a simple tabular environment to align your text nicely 
{\large \bf The Design and Analysis of Algorithms} \\
National Taiwan University, Fall 2020  \\
\hline % \hline produces horizontal lines.
\\
\end{tabular} % Our tabular environment ends here.

\vspace*{0.3cm} % Now we want to add some vertical space in between the line and our title.

\begin{center} % Everything within the center environment is centered.
	{\Large \bf Homework 1} % <---- Don't forget to put in the right number
	\vspace{2mm}
	
        % YOUR NAMES GO HERE
	{\bf Yu-Chieh Kuo B07611039} % <---- Fill in your names here!
		
\end{center}  
\vspace{0.4cm}

This homework answers the problem set sequentially. 

\begin{enumerate}
\item {\it Find an expression for the sum of the i-th row of the following triangle, which is called the Pascal triangle, and prove the correctness of your claim. The sides of the triangle are 1s, and each other entry is the sum of the two entries immediately above it}. % <--- For future Homework sets you of course have to change the questions.

\textbf{Answer:}\\
Let $T(i)$ be the sum of the $i$-th row of the Pascal triangle, we have $T(i)=2^{i-1}$
\emph{<proof>} \\
In the base case, when $i=1$, the sum of the first row is 1, which is equal to $2^{1-1}$. Assume the sum of row $n$ is $2^{n-1}$, then the element of row $n+1$ are each formed by adding two elements of row $n$, and each element of row $n$ contributes to forming two elements of row $n+1$. Thus, the sum of the $n+1$ row is $2 \cdot 2^{n-1} = 2^{n}$ as aquired.\\
By induction, we find the expression for the sum of the $i$-th row of the Pascal triangle.

\item {\it The Harmonic series $H(k)$ is defined by $H(k) = 1+ \frac{1}{2} + \frac{1}{3} +\cdots+ \frac{1}{k-1} + \frac{1}{k}$. Prove that $H(2^n) \geq 1 + \frac{n}{2}$ , for all $n\geq 0$ (which implies that $H(k)$ diverges).}

\textbf{Answer:}\\
In the basic case, when $n=0$, $H(2^0)=H(1)=1\geq1$. Assume that $H(2^n) \geq 1+\frac{2}{n}$ is true, when the case $n+1$, \\
\[\begin{split}
    H(2^{n+1}) & = 1 + \frac{1}{2} + \cdots + \frac{1}{2^n} + \frac{1}{2^n + 1} + \cdots + \frac{1}{2^{n+1}} \\
    & = H(2^n) + \frac{1}{2^n + 1} + \cdots + \frac{1}{2^{n+1}} \\
    & \geq (1+\frac{n}{2}) + \frac{1}{2^n + 1} + \cdots + \frac{1}{2^{n+1}} \\
    & \geq (1+\frac{n}{2}) + 2^n \cdot \frac{1}{2^{n+1}} \\
    & \geq (1+\frac{n}{2}) + \frac{1}{2} \\
    & \geq 1 + \frac{n+1}{2}
\end{split}\]
Since the case $n+1$ is still satisfy the inequality, therefore, we prove the Harmonic series inequality by induction.

\item{\it Consider the following series: 1, 2, 3, 4, 5, 10, 20, 40, ..., which starts as an arithmetic series, but after the first 5 terms becomes a geometric series. Prove that any positive integer can be written as a sum of distinct numbers from this series.}

\textbf{Answer:}\\
Let $P(n)$ be the proposition to carry out the proof. $P(n)$ is true if positive integer $n$ can be written as a sum of distinct numbers from this series and be false if not. \\
When $1\leq n \leq 10$, $P(n)$ is true evidently. Assume $n=k>10$ can be written as a sum of distinct numbers from this series, when $n=k+1$, let $a_i$ be the largest number in series and be less than $k+1$ simultaneously, that means $a_i < k+1$. Also, notice that $a_i > k+1-a_i$ since if $a_i \leq k+1-a_i$, it implies $2a_i = a_{i+1} \leq k+1$ $i.e.$ $a_{i+1}$ the largest number in series and be less than $k+1$ instead of $a_i$, contradicting to the assumption. Besides, $k>k+1-a_i$ implies $P(k+1-a_i)$ is true, which means $k+1-a_i$ can be written as a sum of distinct numbers from this series $a_{x_1}+a_{x_2}+...+a_{x_j}$, and $a_i \notin \{ a_{x_b} | b=1,2,\cdots j \}$. Therefore, $P(k+1)$ is true.\\
By induction, we proof that any positive integer can be written as a sum of distinct numbers from this series.


%All positive integer $N$ can be represented as the following form:
%\begin{center}
%    $N = 10(n) + r,\ where \ n \in N \ and \ 1\leq r \leq 9$
%\end{center}
%In this series, $r$ can be written as a sum of distinct numbers from this series, for example, 6 can be written as 1+5 and 7 can be written as 2+5. Also, the part of $10(n)$ can also be represented as the sum of distinct numbers in the series. Therefore, we proof that any positive integer can be written as a sum of distinct numbers from this series.

\item {\it Consider the recurrence relation for Fibonacci numbers $F(n) = F(n-1)+F(n-2)$. Without solving this recurrence, compare $F(n)$ to $G(n)$ defined by the recurrence $G(n) = G(n-1)+G(n-2)+1$. It seems obvious that $G(n) > F(n)$ (because of the extra 1). Yet the following is a seemingly valid proof (by induction) that $G(n) = F (n) - 1$. We assume, by induction, that $G(k) = F(k)-1$ for all $k$ such that $1\leq k \leq n$, and we consider $G(n + 1) = G(n) + G(n - 1) + 1 = F (n) - 1 + F (n - 1) - 1 + 1 = F (n + 1) - 1$. What is wrong with this proof?}

\textbf{Answer:}\\
When $n=1$, $F(1)=1$ and $G(1) = 1+1=2$. But in the process of induction, we could see that $G(1)=F(1)-1$, but obviously it is wrong since $F(1)=1$ and $G(1) = 2$. Therefore, the proof is wrong with incorrect mathematical induction.

\item {\it The set of all binary trees that store non-negative integer key values may be defined inductively as follows.}
\begin{enumerate}
    \item {\it The empty tree, denoted $\bot$, is a binary tree.}
    \item {\it If $t_l$ and $t_r$ are binary trees, then $node(k,t_l,t_r)$, where $k \in Z$ and $k \geq 0$, is also a binary tree.}
\end{enumerate}
{\it So, for instance, $node(2,\bot,\bot)$ is a single-node binary tree storing key value 2 and $node(2,node(1,\bot,\bot),\bot)$ is a binary tree with two nodes — the root and its left child, storing key values 2 and 1 respectively. Pictorially, they may be depicted as below.}
\begin{enumerate}

    %(a)
    \item {\it Define inductively a function $SUM$ that computes the sum of all key values of a binary tree. Let $SUM(\bot) = 0$, though the empty tree does not store any key value.}
    
\textbf{Answer:}\\
The inductive function $SUM$ is as below: \\
$SUM(node(k,t_l,t_r)) $ = 
\begin{cases}
k + SUM(t_l) + SUM(t_r) & \text{tree is non-empty} \\
0 & \text{tree is $\bot$}
\end{cases}
    
    %(b)
    \item {\it Suppose, to differentiate the empty tree from a non-empty tree whose key values sum up to 0, we require that $SUM(\bot) = -1$. Give another definition for $SUM$ that meets this requirement; again, induction should be used somewhere in the definition.}

\textbf{Answer:}\\
The inductive function $SUM$ is as below: \\
$SUM'(node(k,t_l,t_r)) $ = 
\begin{cases}
k + SUM'(t_l) + SUM'(t_r) & \text{tree is non-empty} \\
0 & \text{tree is $\bot$}
\end{cases}

    %(c)
    \item {\it Define inductively a function $MBSUM$ that determines the largest among
    the sums of the key values along a full branch from the root to a leaf. Let $MBSUM( \bot) =
    0$, though the empty tree does not store any key value}

\textbf{Answer:}\\
The inductive function $MBSUM$ is as below: \\
$MBSUM(node(k,t_l,t_r)) $ \\ = 
\begin{cases}
k+max(MBSUM(t_l),MBSUM(t_r)) & \text{tree is non-empty} \\
0 & \text{tree is $\bot$}
\end{cases} \\
Note that the function $max$ will return the maximum value in parameters.

    %(d)
    \item {\it Suppose, to differentiate the empty tree from a non-empty tree whose
    key values on every branch sum up to 0, we require that $MBSUM (\bot) = −1$. Give
    another definition for $MBSUM$ that meets this requirement; again, induction should
    be used somewhere in the definition. }

\textbf{Answer:}\\
The inductive function $MBSUM$ is as below: \\
$MBSUM(node(k,t_l,t_r)) $ \\ = 
\begin{cases}
k+max(MBSUM'(t_l),MBSUM'(t_r)) & \text{tree is non-empty} \\
-1 & \text{tree is $\bot$}
\end{cases} \\
Note that the function $max$ will return the maximum value in parameters.

\end{enumerate}
\end{enumerate}
\end{document}
